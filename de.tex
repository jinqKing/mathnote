% Created 2023-11-14 Tue 21:28
\documentclass[11pt]{report}
\usepackage{ctex}
\usepackage[utf8]{inputenc}
\usepackage[T1]{fontenc}
\usepackage{graphicx}
\usepackage{grffile}
\usepackage{longtable}
\usepackage{wrapfig}
\usepackage{rotating}
\usepackage[normalem]{ulem}
\usepackage{amsmath}
\usepackage{textcomp}
\usepackage{amssymb}
\usepackage{capt-of}
\usepackage{hyperref}
\author{King}
\date{\today}
\title{微分方程指南}
\hypersetup{
 pdfauthor={King},
 pdftitle={微分方程指南},
 pdfkeywords={},
 pdfsubject={},
 pdfcreator={Emacs 29.1 (Org mode 9.6.6)}, 
 pdflang={English}}
\begin{document}

\maketitle
\tableofcontents



\part{概念总结}
\label{sec:orgbaa0094}
方程表示一种约束关系,它的解就是使方程成立的元素,在微分方程里往往是一个函数,它可能可以表达为显函数,或者只能写为隐函数,甚至于我们只能拿符号代替它的解。
能力有限,还没学过复变函数,所以只会初等的解法。

\chapter{{\bfseries\sffamily STARTED} 目前对于一般的方程最一般的方法为: \textbf{幂级数解法}}
\label{sec:org8f0930e}

文中一般使用 \(x\) 为自变量。
\part{求解析解}
\label{sec:org6cdb1d7}
一般手算思路
线性
\begin{itemize}
\item \(\rightarrow\) 计算齐次方程的特征方程解
\end{itemize}
无重根 \(\rightarrow\) 得到 \uline{齐次通解}
有重根 \(\rightarrow\) 依据重数补齐
\begin{itemize}
\item 非齐次- \_特解\textsubscript{(\(y=y_h+y_p\))}
\begin{itemize}
\item 形式简单
\(\rightarrow\) 待定系数
\item \(\rightarrow\) 常数变易 \(\rightarrow\) 解 Wronski 方程得到特解
\end{itemize}
\end{itemize}

非线性
已知特解:
\(\rightarrow\) 常数变易


重根:
一般方法

d'Alembert /降阶法

\(y_2(x)=v(x)y_1(x)\)
\(y_1\) 为已知形式的解


求出隐式解,那么根据初值条件代入,取最小的可能区间
\chapter{一阶方程}
\label{sec:org807a2d4}
直接积分
积分因子
\section{线性方程}
\label{sec:orgb91fd6f}
形式: \(P(x)y'+Q(x)y+R(x)=0\)
思想:等价变化(乘除)使等式左侧化为乘积微分的结果,和化为积的微分
案例:
\begin{itemize}
\item \((4+x)^2y'+2xy=4x\)
左侧形式特殊: \(((4+x)^2)'=2x\)
化为: \(\frac{\mathrm{d}}{\mathrm{d}x}((4+x)^2y)=4x\)
两边积分: \(y=\frac{2x^2+C}{(4+x)^2}\)
\item \(y'+\frac{1}{2}y=\frac{1}{3}\mathrm{e}^{x/3}\)
乘(尚未知)积分因子:\(\mu y' +\frac{1}{2}\mu y=\mu\frac{1}{3}\mathrm{e}^{x/3}\)
观察左侧得到方程: \(\mu_x=\frac{1}{2}\mu\)
令(只要一个符合的 \(\mu\) 即可) \(\mu=\mathrm{e}^{x/2}\)
得到: \(\frac{\mathrm{d}}{\mathrm{d}x}(\mathrm{e}^{x/2}y)=\frac{1}{3}\mathrm{e}^{x/3}\)
两边积分,通解: \(y=\frac{3}{5}\mathrm{e}^{x/3}+C\mathrm{e}^{-x/2}\)
\item \(2y'+xy=2\)
即 \(y'+\frac{x}{2}y=1\)
\(\mu=\mathrm{e}^{x^2/4}\)
\(\mathrm{e}^{x^2/4}y'+\frac{x}{2}\mathrm{e}^{x^2/4}y=\mathrm{e}^{x^2/4}\) 即 \(\frac{\mathrm{d}}{\mathrm{d}x}(y\mathrm{e}^{x^2/4})=\mathrm{e}^{x^2/4}\)
(积不出来但)得到形式 \(y=\mathrm{e}^{-x^2/4}\int \mathrm{e}^{x^2/4}\mathrm{d}x +C\mathrm{e}^{-x^2/4}\)
\end{itemize}
方法:
条件:(方程)
$$y'+py=q$$
过程:
同时乘以积分因子 \(\mu\)
$$\mu y'+\mu py=q$$
如果可以化为,得到条件
\begin{equation}
\label{eq:32}
\mu_x=p\mu
\end{equation}
于是, \(\frac{\mathrm{d}\mu}{\mu}=p\mathrm{d}x\)
(不论常数)可取
\begin{equation}
\label{eq:33}
\mu=\mathrm{e}^{\int p \mathrm{d}x}
\end{equation}
乘回方程,得到
\begin{equation}
\label{eq:34}
\frac{\mathrm{d}}{\mathrm{d}x}(\mu y)=\frac{\mathrm{d}}{\mathrm{d}x}(\mathrm{e}^{\int p \mathrm{d}x}y)=\mu q
\end{equation}
最终解的形式为
\begin{equation}
\label{eq:35}
y=\mathrm{e}^{-\int p\mathrm{d}x}\int \mu q\mathrm{d}x=\mathrm{e}^{-\int p\mathrm{d}x}\int \mathrm{e}^{\int p \mathrm{d}x} q\mathrm{d}x
\end{equation}

\section{分离变量}
\label{sec:orgebba323}
对于简单的方程,将 x,y 分到方程两边,同时积分
\section{积分因子/恰当方程}
\label{sec:orga35725b}
\begin{enumerate}
\item 恰当方程
\label{sec:orgbfd60e5}
转化为全微分方程。
\begin{equation}
\label{eq:17}
M\mathrm{d}x+N\mathrm{d}y=0
\end{equation}
其中 M,N 关于 x,y 连续且可微
(具有连续的一阶偏导数)

我们希望找到一个恰当的函数 u ,使得 \(M=\frac{\partial u}{\partial x},\quad N=\frac{\partial u}{\partial y}\)
也就是
\begin{equation}
\label{eq:21}
\begin{split}
\frac{\partial u}{\partial x}\mathrm{d}x+\frac{\partial u}{\partial y}\mathrm{d}y &=0\\
\mathrm{d}u&=0
\end{split}
\end{equation}
所以得到隐式解 \(u=C\)

\paragraph{条件}
方程满足:
$$\frac{\partial M}{\partial y}=\frac{\partial N}{\partial x}$$

\paragraph{案例}
\begin{itemize}
\item \((y\cos x+2x\mathrm{e}^{y})+(\sin x+x^2\mathrm{e}^{y}-1)y'=0\)
验证: \(\frac{\partial y\cos x+2x\mathrm{e}^{y}}{\partial y}=\cos x +2x\mathrm{e}^{y}=\frac{\partial \sin x+x^2\mathrm{e}^{y}-1}{\partial x}\)
求解原函数:偏积分
$$u=\int y\cos x+2x\mathrm{e}^{y} \mathrm{d}x=-y\sin x+\mathrm{e}^yx^2+C(y)$$
对另一项(y)偏导
$$\sin x+x^2\mathrm{e}^{y}-1=\frac{\partial u}{\partial y}=-\sin x+x^2\mathrm{e}^y+C'$$
$$C=\int 2\sin x+1 \mathrm{d}y=2y\sin x+y$$
通解: \(-y\sin x+\mathrm{e}^yx^2+2y\sin x+y=A\)
\end{itemize}

\paragraph{特殊形式}
一些全微分
fn
\begin{eqnarray}
\label{eq:36}
yx^{y-1}\mathrm{d}x+x^y\ln x \mathrm{d}y=\mathrm{d}x^{y}\\
\frac{y\mathrm{d}x-xdy}{y^{2}}=d(\frac{x}{y})\\
\frac{y\mathrm{d}x-xdy}{yx}=d(\ln \left| \frac{x}{y} \right|)\\
\frac{y\mathrm{d}x-xdy}{x^{2}+y^{2}}=d(arctan \frac{x}{y})\\
\frac{y\mathrm{d}x-xdy}{x^{2}+y^{2}}=\frac{1}{2}d(\ln \left| \frac{x-y}{x+y} \right|
\end{eqnarray}
\(yx^{y-1}\mathrm{d}x+x^y\ln x \mathrm{d}y=\mathrm{d}x^{y}\)
\(\frac{y\mathrm{d}x-xdy}{y^{2}}=d(\frac{x}{y})\)
\(\frac{y\mathrm{d}x-xdy}{yx}=d(\ln \left| \frac{x}{y} \right|)\)
\(\frac{y\mathrm{d}x-xdy}{x^{2}+y^{2}}=d(arctan \frac{x}{y})\)
\(\frac{y\mathrm{d}x-xdy}{x^{2}+y^{2}}=\frac{1}{2}d(\ln \left| \frac{x-y}{x+y} \right|\)
\item 积分因子
\label{sec:org1fa566e}
区别之前,这里是找到原函数,之前是将方程左侧化为
-可以求解所有方程,但不一定可以解出
不是恰当的方程转化为恰当
把
ref
乘以 \(\mu\)
得到
\begin{equation}
\label{eq:25}
\mu M\mathrm{d}x+\mu N\mathrm{d}y=0
\end{equation}

必要条件
\begin{equation}
\label{eq:23}
\frac{\partial \mu M}{\partial y}=\frac{\partial \mu N}{\partial x}
\end{equation}

找法
假设 \(\mathrm{d}u=\)
\(\mu_y M+\mu M_y=\mu_x N+\mu N_x\)
(不会解方程,化简形式,令某些项为0)找一个只与 x 或 y 有关的积分因子
只与 x 相关的积分因子
\(\mu_xN=\mu(M_y-N_x)\)
\begin{equation}
\label{eq:24}
\frac{1}{\mu}\frac{\mathrm{d}\mu}{\mathrm{d}x}
=\frac{(M_y-N_x)}{N} 
\end{equation}
\begin{equation}
\label{eq:26}
\mu=\mathrm{e}^{\int \frac{(M_y-N_x)}{N} \mathrm{d}x}
\end{equation}

\paragraph{案例}
\begin{itemize}
\item \((3xy+y^2)+(x^2+xy)y'=0\) [完整]
验证发现非恰当: \(\frac{\partial 3xy+y^2}{\partial y}=3x+2y\neq2x+y= \frac{\partial x^2+xy}{\partial x}\)
写出积分因子方程: \(\mu_y M+\mu M_y=\mu_x N+\mu N_x\)
即 \(\mu_y (3xy+y^{2})+\mu (3x+2y)=\mu_x (x^2+xy)+\mu (2x+y)\)
取 \(\mu_y=0\)
得 \(\mu(x+y)=\mu_x(x^2+xy) \rightarrow \mu=x\mu_x\)
于是令 \(\mu=x\)
得恰当方程
M 对 x 偏积分 \(u=\int 3x^2y+xy^2\mathrm{d}x=x^3y+\frac{1}{2}x^2y^2+c(y)\)
u 对 y 偏导 
解为 \(x^3y+\frac{1}{2}x^2y^2=C\)
\end{itemize}
\item 原理
\label{sec:orgdc02fa1}
\begin{enumerate}
\item 恰当方程的判据
\label{sec:org68dbff3}
判断
ref
是(具有连续一阶偏导)
是恰当微分方程(可以转化为 \(\mathrm{d}u=0\))的充分必要条件 $$\frac{\partial M}{\partial y}=\frac{\partial N}{\partial x}$$

ref
具有一般性

充分:
满足条件则能够找到一个 u 使得  \(M=\frac{\partial u}{\partial x},\quad N=\frac{\partial u}{\partial y}\)

找 u
对 M 向 x 积分,取原函数为 u, \(u=\int M \mathrm{d}x +C\)
(C 与 x 必然无关)可以认为  \(u(x,y)=\int M(x,y) \mathrm{d}x +C(y)\)
\sout{如果 C 的确只与 y 有关那么}
(我们希望 \(\mathrm{d}u-\frac{\partial u}{\partial x}\mathrm{d}x=\mathrm{d}C=\mathrm{d}C(y)\) 这个 dC 与 dx 无关,就的确说明找到的函数 u 它的全微分是原方程,便得证)

研究 C ,(无法直接得到 dC 只好先算 u 偏导)将所得结果对 y 求偏导

\(\frac{\partial u}{\partial y}=\frac{\partial\int M \mathrm{d}x}{\partial y}+\frac{\mathrm{d}C}{\mathrm{d}y}=N\)
所以 \(\frac{\mathrm{d}C}{\mathrm{d}y}=N-\frac{\partial}{\partial y}\int M \mathrm{d}x\)

证明它与 x 无关
于是它就是全微分
,使用偏导为0验证
\begin{equation}
\label{eq:22}
\begin{split}
\frac{\partial}{\partial x}[N-\frac{d}{\mathrm{d}y}\int M \mathrm{d}x] &=\frac{\partial}{\partial x}N-\frac{\partial}{\partial x}\frac{\partial}{\partial y}\int M \mathrm{d}x\\
&=\frac{\partial N}{\partial x}-\frac{\partial}{\partial y}\frac{\partial}{\partial x}\int M \mathrm{d}x \\
&=\frac{\partial N}{\partial x}-\frac{\partial M}{\partial y}=0
\end{split}
\end{equation}
于是 dC 与 x 无关,得证 u 的全微分是原方程。
x,y 对称,证明具有一般性。

必要:
如果是恰当方程那么已经存在函数 u ,它的全微分是原方程,那么
可以
根据
Cla
克莱罗定理
fn
验证两者相等
\end{enumerate}
\end{enumerate}

\chapter{高阶方程}
\label{sec:orgd537f32}
\chapter{二阶方程}
\label{sec:orgf4ae2f8}
以下方法,
齐次方程的特解为0,只需要考虑非齐次情况
\section{特征方程:通解}
\label{sec:org1849277}
试用范围:齐次方程求通解
原理:
\paragraph{具体过程}
首先使用 \(\mathrm{e}^{\lambda x}\) 试探方程的解,得到代数方程,即它的特征方程。使用代数方程解得 \(\lambda\) 得到通解
\paragraph{特殊情况}
如果出现重根,
相当于少了一些解,因此我们需要补齐至方程阶数的解

,那么可以猜测这些需要添加的解形式和 \(\mathrm{e}^{\lambda x}\) 类似,可以尝试发现 \(x\mathrm{e}^{\lambda x}\) 满足方程,
则根据重数使用如 \(x\mathrm{e}^{\lambda_{p} x},\cdots,x^{(p-1)}\mathrm{e}^{\lambda_{p}x}\) 的形式扩展解的数目,补满 n 个,假设  \(\lambda_p\) 有 p 重根,就添加额外 p-1 个解如前。
\section{积分因子/二阶恰当方程}
\label{sec:org4aa3c2c}
\begin{enumerate}
\item {\bfseries\sffamily TODO} 案例
\label{sec:orgc12b679}

\item 原理
\label{sec:orga73971e}
方程为
\begin{equation}
\label{eq:14}
P(x)y''+Q(x)y'+R(x)y=0
\end{equation}
希望转化为(一阶恰当方程的形式 \(M\mathrm{d}x+N\mathrm{d}y=0\))
\begin{equation}
\label{eq:15}
(P(x)y')'+(f(x)y)'=0
\end{equation}
可知
\begin{equation}
\label{eq:18}
P'y'+Py''+f'y+fy'=0
\end{equation}
条件为
\begin{align*}
P'+f'=Q\\
f'=R
\end{align*}
得到条件
\begin{equation}
\label{eq:19}
P''-Q'+R=0
\end{equation}
实际代入计算
\begin{equation}
\label{eq:20}
f=Q-P'
\end{equation}
\end{enumerate}

\section{{\bfseries\sffamily TODO} 换自变量}
\label{sec:org8e2f4d4}
Euler 方程
(\ref{eq:EulerEqu})
一般形式

\chapter{求非齐次特解}
\label{sec:org19965e6}

\section{待定系数法}
\label{sec:orgba6f55e}
\paragraph{概要}
非齐次方程观察右端项,猜测可能形式(采取相同相似形式),代入解出系数。

需要考虑重根和与齐次通解重复的部分。如果有重复。

\paragraph{举枚}
\begin{itemize}
\item \(y''-3y'-4y=3\mathrm{e}^{2x}\)
\end{itemize}
通解 \(\mathrm{e}^{-x},\mathrm{e}^{4x}\)
猜特解形式 \(Y_p=A\mathrm{e}^{2x}\)
代入
\begin{equation}
\label{eq:6}
4A+6A-4A=3
\end{equation}
通解为 \(y=\frac{1}{2}\mathrm{e}^{2x}+C_1\mathrm{e}^{-x}+C_2\mathrm{e}^{4x}\)

\begin{itemize}
\item \(y''-3y'-4y=3\mathrm{e}^{2x}\)
原特解与齐次通解相同,特解取 \(Y_p=Ax\mathrm{e}^{2x}\) 解得 \(A=-\frac{2}{5}\)

\item \(y''-3y'-4y=3\mathrm{e}^{2x}+2\sin x-8\mathrm{e}^x\cos 2x\)
分别计算

\item \(y'''-4y'=x+3\cos x+\mathrm{e}^{-2x}\)
齐次通解: 0,2,-2
\(g(x)=x+3\cos x+\mathrm{e}^{-2x}\)
\begin{enumerate}
\item 与0重  \(Y_{p1}=x(A_0+A_1x)\)
\item \(Y_{p2}=B\cos x+C\sin x\)
\item 与-2重 \(Y_{p3}=Ex\mathrm{e}^{-2x}\)
\end{enumerate}

\item \(y^{(4)}+2y''+y=3\sin x-5\cos x\)
齐次通解: \((r^2+1)^2=0\Rightarrow r=\pm \mathrm{i}\)
\(y_h=\cos x+\sin x+x\cos x+x\sin x\)
特解: \(Y_p=x^2(A\cos x+B\sin x)\)
\end{itemize}

\section{常数变易法}
\label{sec:orgf64326a}
使用已经解得的通解,视常数为含自变量的函数,代入方程继续求解。

\textbf{简化思路:[核心]}有 n 个系数需要确定,所以需要 n 方程,可以自由取这些方程。

\paragraph{技巧}
使用 Wronski 行列式


\begin{enumerate}
\item {\bfseries\sffamily TODO} 为什么
\label{sec:org8fdf6a1}

约束方程,令所有 \(c'_iy^n_i=0\) ,本质上是代入方程后使每次求导让系数求导的项为0。

最后结果是 Wronski 行列式

\begin{eqnarray*}
\label{eq:2}
\sum c'_iy_i^1=0\\
\sum c'_iy_i^2=0\\
\cdots\\
\sum c'_iy_i^{n-1}=0
\end{eqnarray*}
可化为:

\begin{equation}
\label{eq:5}
\begin{pmatrix}
y_1&y_2&\cdots&y_n\\
y'_1&y'_2&\cdots&y'_n\\
\vdots&&\cdots&\vdots\\
y^{(n-1)}_1&y^{(n-1)}_2&\cdots&y^{(n-1)}_n
\end{pmatrix}
\begin{pmatrix}
c_1\\c_2\\\vdots\\c_n
\end{pmatrix}
=
\begin{pmatrix}
0\\0\\\vdots\\0
\end{pmatrix}
\end{equation}

之后利用 Crammer 法则求解 \(c_i=\frac{|W(i)|}{|W|}\) 。
\end{enumerate}
\section{归零因子法(A)}
\label{sec:org5d31132}
\paragraph{概要}
化非齐次方程为齐次方程
消去方程右边项
代价是方程阶数升高
\paragraph{举例}
使用 \(D\) 微分算子
\begin{itemize}
\item \((D-2)^2(D+1)y=3\mathrm{e}^{2x}-x\mathrm{e}^{-x}\)
右侧第一项 \(\mathrm{e}^{2x}\) : D-2
第二项 \(x\mathrm{e}^{-x}\) : \((D+1)^2\)
于是归零因子为: \((D+1)^2(D-2)\)
两边同乘归零因子得到: \((D+1)^3(D-2)^4y=0\)
就解7阶齐次方程,注意可能有增根
7阶通解: \$y\textsubscript{h}=(c\textsubscript{1})
\end{itemize}

\chapter{幂级数解法}
\label{sec:org591afbe}

\section{幂级数}
\label{sec:org99d5f04}
形式: $$\sum_{i=0}^{\infty}=a_n(x-x_0)^{n}$$
原理定理保证: ref
\paragraph{收敛半径的计算}
\begin{itemize}
\item 比值
\begin{equation}
\label{eq:3}
\lim_{n\rightarrow\infty}\left| \frac{a_{n+1}(x-x_0)^{n+1}}{a_n(x-x_0)^{n}} \right|=(x-x_0)\lim_{n\rightarrow\infty}\left| \frac{a_{n+1}}{a_n} \right|
\end{equation}
\item 根值
\begin{equation}
\label{eq:11}
\lim_{n\rightarrow\infty}\sqrt[n]{\left| a_n(x-x_0)^{n} \right|}=(x-x_0)\sqrt[n]{\left| a_n \right|}
\end{equation}
\end{itemize}

变换求和哑元,保证 x 的次方相同于是可以把 x 约掉,只解出系数的递推关系

如果 x 次方相同,仇和其实不一样,把多出来的项取出来,令它们为0,使得下标相同,去掉求和,得递推。
\paragraph{性质}
条件:收敛级数, \(f(x)=\sum a_n(x-x_0)^n\)

\begin{itemize}
\item 可加减乘除(保持收敛)
\item 连续的 \(f(x)\) 在收敛区间上有各阶导数。
\item 是解析的 \(\rightarrow\) 可泰勒展开
\item 两个级数相等,对应次方的系数相等
\end{itemize}

$$P(x)y''+Q(x)y'+R(x)y=0$$  
P Q R
多项式 \(\rightarrow\) 解析的

常点 : ordinary point
\(P(x_0)\)

Wronski
\(W[Y,y](x_0)\)

\begin{equation}
\label{eq:16}
y''+k^2y=0
\end{equation}


\section{特殊方程}
\label{sec:orge6ad9c6}
\begin{enumerate}
\item Airy
\label{sec:org1d82057}

\item Legrendre
\label{sec:org8816c69}
\item Berssel
\label{sec:orgcfc9a73}
\end{enumerate}
\section{Euler 方程}
\label{sec:org6900c72}
形式:
\begin{equation}
\label{eq:EulerEqu}
x^2y''+\alpha xy' +\beta y=0,\quad x>0
\end{equation}
引入中间变量 x

$$x=\mathrm{e}^t,\quad t=\ln x$$
于是各项关系为
\begin{equation}
\label{eq:9}
\frac{\mathrm{d}t}{\mathrm{d}t}=\frac{1}{x},\quad \frac{\mathrm{d}x}{\mathrm{d}t}=\mathrm{e}^t \rightarrow \frac{\mathrm{d}y}{\mathrm{d}x}=\frac{\mathrm{d}y}{\mathrm{d}t}\frac{\mathrm{d}t}{\mathrm{d}x}=\frac{1}{x}\frac{\mathrm{d}y}{\mathrm{d}t}
\end{equation}

$$\frac{\mathrm{d}^2y}{\mathrm{d}x^2}=\frac{\mathrm{d}}{\mathrm{d}x}(\frac{1}{x}\frac{\mathrm{d}y}{\mathrm{d}t})=-\frac{1}{x^2}\frac{\mathrm{d}y}{\mathrm{d}t}+\frac{1}{x}(\frac{\mathrm{d}t}{\mathrm{d}x}\frac{\mathrm{d}x}{\mathrm{d}t})\frac{\mathrm{d}}{\mathrm{d}x}\frac{\mathrm{d}y}{\mathrm{d}t}=\frac{1}{x^2}(\frac{\mathrm{d}^2y}{\mathrm{d}t^2}-\frac{\mathrm{d}y}{\mathrm{d}t})$$
代入得到
\begin{equation}
\label{eq:10}
\frac{\mathrm{d}^2y}{\mathrm{d}x^2}-\frac{\mathrm{d}y}{\mathrm{d}x}+\alpha \frac{\mathrm{d}y}{\mathrm{d}x}+\beta y=0
\end{equation}
\paragraph{案例}
\begin{enumerate}
\item 
\end{enumerate}
\chapter{Laplace 变换求解}
\label{sec:orgf46d000}

\part{性质定理}
\label{sec:org98e4bb2}
\chapter{形式定义}
\label{sec:org44eb8f5}
微分算子:
\begin{equation}
\label{eq:7}
L[y]:= \sum_i^n a_i y^{(i)} =
\end{equation}
特征多项式:
\begin{equation}
\label{eq:8}
Z(r):=\sum_i^n a_i r^i
\end{equation}
特征方程:

\begin{gather}
\label{eq:12}
Z(r)  =  0\\
a_n(r-r_1)^{s_1}\cdots(r-r_k)^{s_k}(r-(\lambda_1+\mathrm{i}\mu_1))^{\tau_1}(r-(\lambda_1-\mathrm{i}\mu_1)^{\tau_1}\cdots(r-(\lambda_l\pm\mathrm{i}\mu_l)^{\tau_l}=0
\end{gather}

阶数
\begin{equation}
\label{eq:13}
n=Deg \quad Z=s_1+\cdots+s_k+2(\tau_1+\tau_l)
\end{equation}

\chapter{存在唯一性定理}
\label{sec:org4ee3f0c}
{\bf 多项式、代数基本定理 }

将一般的 \(f(x,y^{(n)},\cdots,y)=0\) 作因式分解如下:

\begin{eqnarray}
\label{eq:1}
a_ny^{(n)}+ a_{n-1}y^{(n-1)}+\cdots+a_1y' a_0y& = & 0 \nonumber\\
a_n \frac{\mathrm{d}^ny}{\mathrm{d}x^n} + a_{n-1}\frac{\mathrm{d}^{n-1}y}{\mathrm{d}x^{n-1}}+\cdots+a_1 \frac{\mathrm{d}y}{\mathrm{d}x}+a_0y & = & 0 \nonumber\\
(b_1 \frac{\mathrm{d}}{\mathrm{d}x}+c_1)(b_2 \frac{\mathrm{d}}{\mathrm{d}x}+c_2)\cdots(b_n \frac{\mathrm{d}}{\mathrm{d}x}+c_n)y & = & 0
\end{eqnarray}

可以分解因式由
?保证,
然后可以利用代数基本定理确保我们必定得到。

直观上看,取最右边与 y 乘的项,得到为0,于是(每一项可以交换下)
\begin{equation}
\label{eq:31}
(b_i \frac{\mathrm{d}}{\mathrm{d}x}+c_i)y  =  0
\end{equation}
可以解出 n 个解。
(同时可以从一个角度说明使用 \(\mathrm{e}^{\lambda x}\) 尝试方程的合理性:
\(\frac{\mathrm{d}y}{\mathrm{d}x}=-\frac{c_i}{b_i}y\) 可以解出指数形式解)
fn 完备

线性方程:整体
非线性:局部解

{\bf 皮卡序列}

核心:把初值条件代入反复积分函数叙列趋近

\paragraph{方法}
条件:
初值田间

步骤
\begin{enumerate}
\item 代入初值
\item 积分
\end{enumerate}
$$\varphi_{n+1}=\int\limits_0^tf(s,\varphi_n(s))\mathrm{d}s$$   

\paragraph{案例}

Lipschitz 条件
存导函数有界

偏导拆连续且存在
? Langrange 中值定理


先正收敛,再唯一

反例?

\chapter{线性性质}
\label{sec:orgfe36170}
行列式

\section{叠加原理}
\label{sec:org3d96926}
\chapter{Wronski 行列式判断线性相关性}
\label{sec:org6ce8119}
\section{{\bfseries\sffamily TODO} }
\label{sec:orgb07d29a}
现在假设解得齐次(通)解 \(y_1,\cdots,y_n\) ,代入方程得到
\begin{eqnarray*}
a_ny_{1}^{(n)}+ a_{n-1}y_{1}^{(n-1)}+\cdots+a_1y_{1}'+a_0y_{1} & = & 0 \nonumber\\
\cdots\\
a_ny_{n}^{(n)}+ a_{n-1}y_{2}^{(n-1)}+\cdots+a_1y_{n}'+a_{0}y_{n} & = & 0 \nonumber\\
\end{eqnarray*}
共 n 个方程。它们系数为原微分方程系数万们可以用矩阵改写为 \(Ax=0\) 的形式
\begin{equation}
\label{eq:4}
\begin{pmatrix}
y_1&y_2&\cdots&y_n\\
y_1'&y_2'&\cdots&y_n'\\
\vdots&&\cdots&\vdots\\
y^{(n-1)}_1&y^{(n-1)}_2&\cdots&y^{(n-1)}_n\\
\end{pmatrix}^T
\begin{pmatrix}
a_0\\a_1\\\vdots\\a_n
\end{pmatrix}
=
\begin{pmatrix}
0\\0\\\vdots\\0
\end{pmatrix}
\end{equation}

因此根据线性方程的性质,我们要求 \(|A|=0\) (理由是 \(a_0,\cdots,a_n\) 不全为0,所以 \(A\) 存在线性相关部分),即
\begin{equation}
\label{eq:wronski=0}
\begin{vmatrix}
y_1&y_2&\cdots&y_n\\
y_1'&y_2'&\cdots&y_n'\\
\vdots&&\cdots&\vdots\\
y^{(n-1)}_1&y^{(n-1)}_2&\cdots&y^{(n-1)}_n\\
\end{vmatrix}
=0
\end{equation}

此即 Wronski 行列式,可以方便地在微分方程求解中判断解是否线性相关。

线性相关原始定义为:存在线性组合结果为0。即一般的函数应当使用如下判据:
\(c_1y_1+\cdots+c_ny_n=0\) 当且仅当 \(c_1=c_2=\cdots=c_n=0\) 

\begin{verbatim}
如果不是微分方程的解,使用 Wronski 行列式可能会有问题如:
(此处存疑)
\end{verbatim}

\chapter{Abel 定理}
\label{sec:org0562be1}
效果:不需要解方程就可以得到 Wronski 行列式

\chapter{幂级数收敛}
\label{sec:org5fa66db}

\chapter{奇点}
\label{sec:org0324b7d}
找
正则
每一项是有限的

无穷远
意义?

\part{求数值解}
\label{sec:orgfcd1aec}

\chapter{Euler 方法(欧拉折线)}
\label{sec:org6cc2eec}
适用:初值条件问题 IVP

原理:解的唯一性

思想:直线近似,缩小步长逼近

效果:一阶?

方法:

\begin{itemize}
\item 已有条件:
\end{itemize}
\begin{gather}
\label{eq:29}
\frac{\mathrm{d}y}{\mathrm{d}x}=f(x,y)\\
y(x_0)=y_0
\end{gather}
(初值条件)

\begin{itemize}
\item 过程:
取步长 \(h\)
从方程得到在 \((x_{0},y_0)\) 的导数 \(y'=f(x_0,y_0)\)
使用直线方程 \(y=f(x_0,y_0)(x-x_0)+y_0\)
代入 \(x_0+h\) 计算得到 y 重复

\item 公式
\begin{equation}
\label{eq:30}
y_{n+1}=y_n+fh
\end{equation}
\end{itemize}
性质:
收敛性
略
\#(不会)
\end{document}